% DOCUMENT
\documentclass[a4paper]{article}

% PACKAGES
\usepackage{titlesec}
\usepackage[english]{babel}
\usepackage{blindtext}
\usepackage{graphicx} 
\usepackage{enumitem}
\usepackage{amsmath}
\usepackage{mathptmx}
\usepackage{layout}
\usepackage{geometry}

% SETTINGS
% Title Format
\titleformat*{\section}{\large\MakeUppercase}
\titleformat*{\subsection}{\large\MakeUppercase}
\titleformat*{\subsubsection}{\large}
% Remove paragraph indentation
\setlength\parindent{0pt}
% Set margins
\geometry{margin=1in}

% TITLE
\title{\Huge{Control of Electric Motors}}
\author{Faizudeen Kajogbola} 
\date{} %BLACK DATE TO  OMIT DATE FROM TITLE

% DOCUMENT
\begin{document}%\layout
% DISPLAY TITLE
\maketitle

\setlength{\headsep}{5pt}
% \setlength{\voffset}{-0.75in}   % REMOVE EXCESSIVE SPACE AT TOP OF PAGE

% BASIC LAWS AND RELATIONSHIPS
\section{Basic Laws and Relationships}

%---------------------------------------------------------------------------------------------------------------------------%
\subsection{Ampere's Law}
Ampere's law gives the relationship between a magnetic field and the current that produces the magnetic field.
Let's consider a rectangular core of a ferromagnetic material (e.g. Iron). 
Due to high relative permeability of ferromagnets, the produced magnetic field remains in the core.
Ampere's Law in this case then states: 

$$\oint \vec{H} \cdot \vec{dl}= I_{net}$$
- $\vec{H}$ is the magnetic field intensity.\\
- $\vec{dl}$ is the differential element of length along the path of integration.  \\
- $I_{net}$ is the magnetomotive force (i.e. current through coil * no of turns).  \\

Further,
$$\vec{B}= \mu \  \vec{H}$$
- $\mu$ is permeability of the material.  \\

And,
$$\phi= \int_{A} \vec{B} \cdot \vec{dA}$$

An analogy of the Ohm's Law in magnetic circuits is given as:
$$mmf= \frac{\phi}{R_{m}}$$
- $mmf$ is the Magnetomotive Force measured in Ampere-turns.  \\
- $R_m$ is magnetic reluctance.  \\

%---------------------------------------------------------------------------------------------------------------------------%
\subsection{Faraday's Law}
States that EMF is generated in a conductor when it cuts through magnetic flux according to the following relationship:
$$e= -\frac{d\phi}{dt}$$
\textit{the negative sign is explained by Lenz's Law}. \\
If we consider a magnetic field with evenly distributed flux (i.e. constant $\vec{B}$), 
then $\frac{d\phi}{dt}$ becomes $\frac{dA}{dt} \times \vec{B}$. Also, assuming the conductors are 
of a fixed length (which is almost always true) we have:
$$e= \left( \vec{v} \times \vec{B} \right) \cdot \vec{l} $$
- $\vec{v}$ is the velocity at which the conductor is moving relative to the coil.\\
- $\vec{l}$ is the length of the conductor, pointing along the direction that makes the smallest angle with 
$\left( \vec{v} \times \vec{B} \right)$. \\ 

%---------------------------------------------------------------------------------------------------------------------------%
\subsection{Lenz's Law}
States that the direction of EMF induced in a conductor under the influence of a changing magnetic field is 
such that it produces a flux that opposes its causal flux. This accounts for the negative sign in Faraday's Law.  

%---------------------------------------------------------------------------------------------------------------------------%
\subsection{Lorentz Law}
The force experienced by a current-carrying conductor under the influence of a magnetic field is given by:
$$ \vec{F}= \left( \vec{l} \times \vec{B} \right) \ i $$
- $\vec{l}$ is the vector of conductor length pointing along the direction of current flow.


% DC MOTORS
%---------------------------------------------------------------------------------------------------------------------------%
\section{DC Motors}

%---------------------------------------------------------------------------------------------------------------------------%
\subsection{Modelling of DC Motors}
The most common DC motors are seperately excited DC motors. 
They have seperate DC supplies for the armature and field windings.
Modelling a seperately excited DC motor is a 4-step process.

\begin{itemize}

    \item \textbf{Armature Circuit}: 
            The armature circuit can be represented with the following equation:
            $$ V_{a}= i_{a} R_{a} + L\frac{di_{a}}{dt} + e_{b} $$
            - $V_{a}$ is the supply voltage to the armature circuit.\\
            - $R_{a}$, $L_{a}$, and $i_{a}$ are the armature winding resistance, winding inductance and current respectively.\\
            - $e_{b}$ is the back EMF.

    \item \textbf{Back EMF}:
            The back EMF is the Faraday's law on the armature windings as they cut through the field flux. 
            As an extension of Faraday's law, 
            $$e_{b}= k_{b} \omega_{m} \phi_{f} $$ 
            where $k_{b}$ is a constant that depends on the motor design, $\omega_{m}$ is the angular velocity of the rotor, 
            and $\phi_{f}$ is the field flux.

    \item \textbf{Torque}:
            According to Lorentz Law, force is generated on the armature winding conductors as they are carrying currents and under 
            the influence of the field flux. By mounting these conductors on an axle that's free to rotate and due to the effects of 
            commutation (which prevents oscillatory rotation), a torque is developed by the DC motor. Because the direction of armature 
            current is always perpendicular to the direction of field flux in a DC motor, we can express the force on each winding 
            conductor as a multiplication (not a cross product) $\vec{F}= l B i$ in the appropriate direction. 
            For each winding conductor, a torque $\vec{T}= \vec{F} \times \vec{r} $ is produced. If the armature winding coils are 
            fixed in space; designed to be equidistant; and of uniform diameter and length, the equivalent torque generated can be given as: 
            $$ T_{m}= k_{t} i_{a} \phi_{f} $$
            Where $k_{t}$ is a constant that depends on the design of the motor.\\
            In SI units, $k_{b}$ and $k_{t}$ are numerically equal.

    \item \textbf{Mechanical Load System}:
            The mechanical load system of a DC motor consists of the motor's mechanical representation and the load attached 
            (maybe via a set of gears). From Newton's second law, we have:
            $$ T_{m}= J\frac{d \omega_{m}}{dt} + D \omega_{m} + T_{l} $$
            - $J$ is the moment of inertia of the motor.\\
            - $D$ is the coefficeint of rotational friction of the motor.\\
            - $T_{l}$ is the load torque.\\

\end{itemize}

%---------------------------------------------------------------------------------------------------------------------------%
\subsubsection{Aside: Gears}

%---------------------------------------------------------------------------------------------------------------------------%
%---------------------------------------------------------------------------------------------------------------------------%
\subsection{Steady-State Characteristics of DC Motors}

At steady state $\frac{di_{a}}{dt}$ = $\frac{d\omega_{m}}{dt}$ = $0$, the model equations can then be reduced to:

$$ i_{a}= \frac{V_{a} - e_{b}}{R_{a}} $$
$$ T_{m}= \frac{\left(V_{a} - k_{b} \omega_{m} \phi_{f}\right) k_{t} \phi_{f}}{R_{a}} $$
Since $k_{b}$ = $k_{i}$ (assuming SI units), let $K:= k_{b}= k_{t}$
$$ T_{m}= \frac{ K \phi_{f} V_{a} }{R_{a}} - \frac{ K^{2} \phi_{f}^{2} \omega_{m} }{R_{a}} $$
$$ T_{l}= \frac{ K \phi_{f} V_{a} }{R_{a}} - \frac{ \omega_{m} \left( K^{2} \phi_{f}^{2} - D R_{a} \right) }{R_{a}} $$
$$ \omega_{m}=  \frac{ R_{a} }{ \left( K^{2} \phi_{f}^{2} - D R_{a} \right) }  \left( \frac{ K \phi_{f} V_{a} }{R_{a}} - T_{l} \right) $$
For simplicity, ket us ignore the rotational friction in the motor, then we have:
$$ \omega_{m}=  \frac{V_{a}}{ K \phi_{f} } - \frac{ R_{a} }{ \left( K^{2} \phi_{f}^{2} \right) } T_{l}  $$

This shows that the terminal voltage $V_{a}$ and the field flux $\phi_{f}$ are directly and inversely proportional to the motor speed 
$w_{m}$ respectively. Thus to control the motor speed, we can vary $V_{a}$ or $\phi_{f}$. This relationship is dipicted in Figure \ref{fig:DC_motor_speed_torque}.
\\
\begin{figure}
        \centering
        \includegraphics[scale=0.3]{img/DC_motors_speed-torque.png}
        \caption{DC Motor Speed-Torque Relationship \cite{kim17}}
        \label{fig:DC_motor_speed_torque}
\end{figure}

%---------------------------------------------------------------------------------------------------------------------------%
\subsubsection{Armature Voltage Control}
Increasing the terminal voltage of a DC motor (at constant load torque) increases its back EMF and subsequently its speed. 
This is called *terminal Voltage Control*. Since the terminal voltage cannot be raised above the rated voltage of the machine, 
the speed of the motor cannot be increased beyond its rated speed using this method. 
Figure \ref{fig:DC_motor_vol} shows the relationship between the terminal voltage, speed, and torque of a DC motor.
\\
\begin{figure}
        \centering
        \includegraphics[scale=0.3]{img/DC_motors_armature_voltage_control.png}
        \caption{DC Motor terminal Voltage Control \cite{kim17}}
        \label{fig:DC_motor_vol}
\end{figure}

%---------------------------------------------------------------------------------------------------------------------------%
\subsubsection{Field Flux Control}
The field flux of a DC motor is inversely proportional to the motor speed, thus reducing the field flux increases the speed. 
With this method, the motor speed can be raised above the rated value. 
However, the response is often slow because the field windings are designed to produce large flux which often cna't be changed 
rapidly. Also, very low flux can easily be affected by armature reaction.
The typical relationship between the field flux, motor torque and motor speed are shown in Figure \ref{fig:DC_motor_flux}.
\\
\begin{figure}
        \centering
        \includegraphics[scale=0.3]{img/DC_motors_field_flux_control.png}
        \caption{DC Motor Field Flux Control \cite{kim17}}
        \label{fig:DC_motor_flux}
\end{figure}

%---------------------------------------------------------------------------------------------------------------------------%
\subsubsection{Operation Regions of DC Motors}

There are two operating regions for a typical DC motors: *constant torque region* and *constant power region*. \\

In the constant torque region, the motor speed is $0 \le \omega_{m} \le \omega_{base}$. 
It is called the constant torque region because the torque developed is dependent on the armature current (since the field flux is constant). 
The same armature current produces the same torque regardless of the speed. 
$\omega_{base}$ is the speed developed at the rated voltage (may be different from the rated speed depending on the operating conditions).\\
\\
In the constant power (field weakening) region, the speed  $\omega_{m} \ge \omega_{base}$ is increased by reducing the field flux.
The back EMF $e_{b}= k_{b} \phi_{f} \omega_{m} $  can be kept constant as the speed is increasing and the flux is reducing leading to constant armature current.
It is called the constant power region because the terminal voltage and the armature current are constant, yielding a constant power regardless of the motor speed.\\
\\
The speed-torque characteristics of a typical DC motor highlighting the various operating modes is shown in Figure 
\ref{fig:DC_motor_cap}, it is commonly known as the capability curve.

\begin{figure}
        \centering
        \includegraphics[scale=0.3]{img/DC_motor_capability_curve.png}
        \caption{DC Motor Capability Curve \cite{kim17}}
        \label{fig:DC_motor_cap}
\end{figure}


%---------------------------------------------------------------------------------------------------------------------------%
%---------------------------------------------------------------------------------------------------------------------------%
\subsection{Transient Response Characteristics of DC Motors}

%---------------------------------------------------------------------------------------------------------------------------%
\subsubsection{Transfer Function}

By taking the Laplace transform of the model equations, we can obtain the transfer function of the DC motor:

$$ I_{a}(s)= \frac{ V_{a}(s) - E_{b}(s) }{ s L_{a} + R_{a} } $$

$$ \omega_{m}(s)= \frac{ T_{m}(s) - T_{l}(s) }{ s J + D }  $$


\begin{figure}
        \centering
        \includegraphics[scale=0.3]{img/DC_motor_transfer_function.png}
        \caption{DC Motor Transfer Function \cite{kim17}}
        \label{fig:DC_motor_trns}
\end{figure}


$$  \omega_{m}(s)=  \frac{ 
                k_{t} \phi_{f} 
        }{ 
                \left( k_{t} k_{b} \phi_{f}^{2} \right) + \left( s L_{a} + R_{a} \right) \left( s J + D \right)  
        } V_{a}(s) - 
        \frac{ s L_{a} + R_{a} }{ 
                \left( k_{t} k_{b} \phi_{f}^{2} \right) + \left( s L_{a} + R_{a} \right) \left( s J + D \right)  
         } T_{l}(s)
$$

Transfer function from the motor input (armature voltage $V_{a}$) to its output (motor speed $\omega_{m}$) is:

$$  \omega_{m}(s)=  \frac{ 
                k_{t} \phi_{f} 
        }{ 
               s^{2} L_{a} J + s \left( L_{a} D + J R_{a} \right) +  R_{a} D + k_{t} k_{b} \phi_{f}^{2}
        } V_{a}(s) 
$$

$$  \omega_{m}(s)=  \frac{ 
                \frac{ k_{t} \phi_{f} }{ L_{a} J } 
        }{ 
               s^{2} + s \left( \frac{D}{J} + \frac{R_{a}}{L_{a}} \right) +  \frac{R_{a} D}{L_{a} J} +
               \frac{k_{t} k_{b} \phi_{f}^{2}}{L_{a} J}
        } V_{a}(s) 
$$

If we assume $D=0$ and $k_{t}=k_{b}$ such that $k_{t}\phi_{f}= k_{b}\phi_{f}= K$, then we have:

$$
\omega_{m}(s)= \frac{ 
                        \frac{K}{L_{a} J}
                }{
                        s^{2} + s \frac{R_{a}}{L_{a}} + \frac{K^{2}}{L_{a} J}
                }=
                \frac{
                        \frac{1}{K} \frac{R_{a}}{L_{a}} \frac{K^{2}}{J R_{a}}
                }{
                        s^{2} + s \frac{R_{a}}{L_{a}} + \frac{K^{2}}{L_{a} J}
                } V_{a}(s) 
$$

If we define the electric time constant as $\tau_{e}= \frac{L_{a}}{R_{a}}$ and the electromagnetic time constant 
as $ \tau_{m}= \frac{J R_{a}}{K^{2}} $, then our transfer function can be rewritten as:

$$
\frac{\omega_{m}(s)}{V_{a}(s)}= \frac{
                        \frac{1}{K} \frac{1}{\tau_{e} \tau_{m}}
                }{
                        s^{2} + s \frac{1}{\tau_{e}} + \frac{1}{\tau_{e} \tau_{m}}
                }  
$$

%---------------------------------------------------------------------------------------------------------------------------%
\subsubsection{Stability and Nature of Response}

From the derived transfer function, we can extract the poles of the system as:

$$
s_{1,2}= -\frac{1}{2 \tau_{e}} \pm \sqrt{\frac{1}{4 \tau_{e}^{2}} - \frac{1}{\tau_{e} \tau_{m}} }
$$

Stability of the system can then be evaluated based on the position of the poles. 
Figure \ref{fig:Stability} shows typical system pole locations and their corresponding transient responses. 

\begin{figure}
        \centering
        \includegraphics[scale=0.3]{img/Stability.png}
        \caption{Typical pole locations and corresponding transient responses \cite{kim17}}
        \label{fig:Stability}
\end{figure}

We can investigate the nature of response of the system by comparing it to the standard second order lag:

$$ G(s)= \frac{ \omega_{n}^{2} }{ s^{2} + 2 \zeta\omega_{n} s + \omega_{n}^{2} } $$

From this, we can extract 
$ \omega_{n}= \frac{1}{\sqrt{\tau_{e}\tau_{m}}} $ 
and $ \zeta= \frac{1}{2} \sqrt{\frac{\tau_{m}}{\tau_{e}}} $

%---------------------------------------------------------------------------------------------------------------------------%
\subsubsection{First Order Simplification}
Further, if we assume that the response of the electrical system is much faster than that of the mechanical system, 
we can ignore the transient of the electrical system such that $L_{a}= 0$, then we have a first order system:

$$
\frac{\omega_{m}(s)}{V_{a}(s)}= \frac{ K }{ R_{a} J s + K^{2} }
$$


%---------------------------------------------------------------------------------------------------------------------------%
%---------------------------------------------------------------------------------------------------------------------------%
\subsection{Motor Control System}

%---------------------------------------------------------------------------------------------------------------------------%
\subsubsection{Cascaded Control Loop}

In controlling a DC motor, we may have a configuration of about 3 cascaded loops. Figure \ref{fig:Cascaded} shows such a setup. 
For topologies like this to work, each inner loop must be considerably faster (at least about 5 times) than it's enclosing loop 
to prevent interference with the control action. \\

\begin{figure}
        \centering
        \includegraphics[scale=0.3]{img/Cascaded_loops.png}
        \caption{Configuration of Motor Control System \cite{kim17}}
        \label{fig:Cascaded}
\end{figure}


%---------------------------------------------------------------------------------------------------------------------------%
\subsubsection{Feedforward Control}

With feedback control, the controller can only react to the disturbance after it must have affected the control output, 
in situations where the disturbance signal can be estimated (e.g. the back EMF of a DC motor), we can proactively couteract its effect 
using a technique known as feedforward control.\\

In DC motor current control, an estimate of the back EMF can be fed-forward. Likewise in speed control of DC motors, an estimate of 
the load torque can be added to the control signal. These compensations help speed up the response of the control system. 

\begin{figure}
        \centering
        \includegraphics[scale=0.3]{img/feedforward_control.png}
        \caption{Feedforward Control \cite{kim17}}
        \label{fig:Cascaded}
\end{figure}


%---------------------------------------------------------------------------------------------------------------------------%
\subsubsection{Aside: Bandwidth and Stability Margins in Feedback Control}

The frequency response of a linear system is system's steady-state open-loop response to sinusoidal inputs of different frequencies. 
The Bode Plot provides the frequency response in terms of the change in magnitude($dB$) and the phase shift($\deg$). 
The gain cross-over frequency is the frequncy at which the magnitude response crossed the $0dB$ line, 
while the phase cross-over frequency is the frequency at which the phase response crosses $180\deg$.  \\

The bandwidth of a closed-loop system is the frequency at which the output power falls to half of the input power, 
that is, the magnitude response is at the $-3dB$ point. 
The bandwidth may be used in measuring the system's speed of response. 
\\
The gain margin of a system refers to how much gain that can be added to a system such that its closed loop response remains stable. 
It is the amount of gain required to push the magnitude response at the phase cross-over frequency to $0dB$.  
The phase margin on the other hand refers to how much phase-shift is a system is allowed within which its closed loop response remains stable. 
It is the amount of phase shift required to push the phase at the gain cross-over frequency to $180\deg$. 
Figure \ref{fig:gain_phase_margin} shows gain and phase margins for stable and unstable systems. 
As a rule of thumb, the gain and phase margins for closed-loop system design should be about $2-10dB$ and $30-60\deg$ respectively.  
\\
\begin{figure}
        \centering
        \includegraphics[scale=0.3]{img/gain_and_phase_margins.png}
        \caption{Gain and Phase Margins \cite{kim17}}
        \label{fig:gain_phase_margin}
\end{figure}

To understand the concept of stability margins in terms of a system's frequency response, let's consider a feedback system with a pure sinusoidal input. 
If the system's frequency response has magnitude $0dB$ and phase $180\deg$ at some frequency $\omega$, 
then the open-loop response of such a system is an equal and opposite sinusoid. 
Thus effectively transforming the negative feedback into a positive feedback and rendering the system unstable. 
Figure \ref{fig:Stability_mar_unst} gives an illustration of this.
\\
\begin{figure}
        \centering
        \includegraphics[scale=0.3]{img/Stability_margins_unstable.png}
        \caption{Unstable Feedback System \cite{kim17}}
        \label{fig:Stability_mar_unst}
\end{figure}


%---------------------------------------------------------------------------------------------------------------------------%
\subsubsection{Current Controller Design}

The block diagram of a DC motor with a PI current controller $D_{i}(s)$ is shown in Figure \ref{fig:DC_current_ctrl}. 
From this, we can extract the system output as: 

$$ I_{a}= \frac{ \frac{D_{i}(s)}{L_{a} s + R_{a}} }{ 1 + \frac{D_{i}(s)}{L_{a} s + R_{a}} } I_{a}^{*}(s) 
        - \frac{ \frac{1}{L_{a} s + R_{a}} }{ 1 + \frac{D_{i}(s)}{L_{a} s + R_{a}} } E_{a}(s) $$

with $D_{i}(s)= K_{pi} + \frac{K_{ii}}{s}$, we have:

$$ I_{a}= \frac{ s K_{pi} + K_{ii} }{ s^{2} L_{a} + s R_{a} + s K_{pi} + K_{ii} } I_{a}^{*}(s) 
        - \frac{1}{ s^{2} L_{a} + s R_{a} + s K_{pi} + K_{ii} } E_{a}(s) $$

since the back EMF pf the motor can easily be estimated from its speed, we can implement a feedforward controller to compensate for this disturbance, 
thus neglecting $E_{a}(s)$, we have:

$$ I_{a}= \frac{ s K_{pi} + K_{ii} }{ s^{2} L_{a} + s R_{a} + s K_{pi} + K_{ii} } I_{a}^{*}(s) $$
\\

\begin{figure}
        \centering
        \includegraphics[scale=0.3]{img/DC_motor+current_controller.png}
        \caption{Current Control \cite{kim17}}
        \label{fig:DC_current_ctrl}
\end{figure}

\textbf{Design with Pole-Zero Cancellation: } We can tune the PI current controller parameters using classical PID tuning methods 
(such Ziegler Nichols, etc), but pole-zero cancellation can also work. 
Let's consider the open loop transfer function:

$$ G_{o}(s)= \frac{ s K_{pi} + K_{ii} }{s} \frac{1}{ s L_{a} + R_{a} } $$

re-arranging, we have: 

$$ G_{o}(s)= K_{pi} \frac{s + \frac{K_{ii}}{K_{pi}}}{s} \frac{\frac{1}{L_{a}}}{s + \frac{R_{a}}{L_{a}}}  $$

now, if we let $\frac{K_{ii}}{K_{pi}} = \frac{R_{a}}{L_{a}}$, then we have:

$$ G_{o}(s)= \frac{K_{pi}}{s L_{a}}  $$

With this, we can evaluate the stability margins for the closed-loop system.  
\\

The resulting closed-loop system has a transfer function: 

$$ G(s)= \frac{K_{pi}}{s L_{a} + K_{pi}}= \frac{K_{pi}}{ L_{a} \left( s + \frac{K_{pi}}{L_{a}} \right) }$$

let $\omega_{ci}= \frac{K_{pi}}{L_{a}}$, then:

$$ G(s)= \frac{\omega_{ci}}{ s + \omega_{ci} } $$

By evaluating $\left| G(j\omega) \right|= \frac{1}{\sqrt{2}}$, we find the bandwidth of the closed loop system to be $\omega_{ci}$. 

Recall that from the zero-pole cancellation design step, we made $\frac{K_{ii}}{K_{pi}} = \frac{R_{a}}{L_{a}}$, 
relating this with our expression for the bandwidth $\omega_{ci}= \frac{K_{pi}}{L_{a}}$, we can select the PI controller parameters as: 

$$ K_{pi}= \omega_{ci} L_{a} $$
$$ K_{ii}= \omega_{ci} R_{a} $$
\\

\textbf{Bandwidth Selection: } In our design process, we have the freedom of selecting the bandwidth of our PI current controller. 
However, we cannot do this arbitrarily. The bandwidth of the PI contoller is limited by 2 major factors: 
the switching frequency of the power electronics that provide the armature voltage; 
and the sampling frequency at which current is fed back to the digital controller. 
As a rule of thumb, the bandwidth should be restricted to $\frac{1}{25}$ of the sampling period. \\

\textbf{Anti-windup Controller: }As with all controllers with an integrator, a PI controller is susceptible to a phenomenon known as integral wind-up. 
This occurs when the integrator keeps accumulating error even after the actuator saturates. 
If the error changes sign suddenly, this accumulated error results in an overshoot and a sluggish respose. 
There are a variety of techniques for combatting integral windup, Figure \ref{fig:back_calculation} shows one of such. 
The idea of the back-calculation anti-windup controller is to compute the difference between the output of the controller and the actual control signal, 
then scale this difference by the inverse of the controller's proportional gain and subtract this from the error signal that's input 
into the integral part of the controller. 

\begin{figure}
        \centering
        \includegraphics[scale=0.3]{img/DC_motor+current_ctrl+anti_windup.png}
        \caption{Current Controller with Back Calculation Anti-windup \cite{kim17}}
        \label{fig:back_calculation}
\end{figure}


%---------------------------------------------------------------------------------------------------------------------------%
\subsubsection{Speed Controller Design}
Because of the cascaded structure of the control system (as shown in Figure \ref{fig:Cascaded}), 
the bandwidth of the closed-loop current control system $\omega_{ci}$ must be significantly higher than that of the speed control system. 
This, together with the sampling frequency at which the speed is fed back to the digital controller are the restricting 
factors on the bandwidth of the speed control system $\omega_{cs}$. 
$\omega_{cs}$ should be selected such that it is within $\frac{1}{10}$ to $\frac{1}{20}$ of the speed sampling frequency. 

Another factor that can greatly simplify our design process is by selecting gains $K_{ps}$ and $K_{is}$ of the PI speed controller 
$D_{s}(s)$ such that $\frac{K_{is}}{K_{ps}}$ is at least 5 times less than $\omega_{cs}$. 

$$ D_{s}(s)= K_{ps} + \frac{K_{is}}{s} $$

\begin{figure}
        \centering
        \includegraphics[scale=0.3]{img/DC_motor_speed_control.png}
        \caption{Speed Control System \cite{kim17}}
        \label{fig:DC_motor_speed_control}
\end{figure}

Figure \ref{fig:DC_motor_speed_control} shows the block diagram of the speed control system (neglecting the rotational damping in the motor). 
We can extract the open-loop transfer function from $\omega_{m}^{*}(s)$ to $\omega_{m}(s)$ as: 

$$ G_{o}(s)= \left( K_{ps} + \frac{K_{is}}{s} \right) \left( \frac{\omega_{ci}}{s + \omega_{ci}} \right)  
                \left( \frac{K_{T}}{s J} \right) $$

with $K_{T}= k_{t} \phi_{f}$.
//

$\omega_{ci}$ >> $\omega_{cs}$ implies that around $\omega= \omega_{cs}$, 

$$ \frac{\omega_{ci}}{s + \omega_{ci}} \approx 1 $$

and $\frac{K_{is}}{K_{ps}}$ << $\omega_{cs}$ means that around $\omega= \omega_{ci}$,

$$ K_{ps} + \frac{K_{is}}{s} \approx K_{ps} $$

working with these simplifying quantifications, the open-loop transfer function around $\omega= \omega_{cs}$ becomes:

$$ G_{o}(s)= \frac{ K_{pi} K_{T}}{s J} $$

yielding a closed-loop transfer function of: 

$$ G(s)= \frac{ \frac{K_{ps} K_{T}}{J} }{ s + \frac{K_{ps} K_{T}}{J} } $$

with $\left| G(j\omega) \right|= \frac{1}{\sqrt{2}}$ at $\omega= \frac{K_{ps} K_{T}}{J}$, 
making $\omega_{cs}= \frac{K_{ps} K_{T}}{J}$

If we make $5 \frac{K_{is}}{K_{ps}}$ = $\omega_{cs}$, then we can select the parameters for the PI speed controller as:

$$ K_{ps}= \frac{J \omega_{cs}}{K_{T}} $$
$$ K_{is}= \frac{K_{ps} \omega_{cs}}{5}= \frac{J \omega_{cs}^{2}}{5 K_{T}} $$


%---------------------------------------------------------------------------------------------------------------------------%
\subsubsection{Power Electronic Converter for DC Motors}

The H-bridge (shown in Figure \ref{fig:H_Bridge}) is the most widely used DC-DC converter for DC motor control. 
There are 2 PWM techniques for adjusting the voltage supplied to the DC motor:

\begin{figure}
        \centering
        \includegraphics[scale=0.3]{img/H-bridge.png}
        \caption{H-Bridge \cite{kim17}}
        \label{fig:H_Bridge}
\end{figure}

\begin{itemize}
        \item \textbf{Bipolar Switching: }In this scheme (shown in Figure \ref{fig:PWM_bipolar}), 
                the diagonal switches ($T_{1}$ and $T_{4}$; $T_{2}$ and $T_{3}$) are toggled at the same time. 
                A triangular carrier wave is used, and the forward conducting switches ($T_{1}$ and $T_{4}$) are triggered 
                on when $V_{a}^{*}$\textemdash which is the desired armature voltage\textemdash is greater then the carrier wave. 
                The instanteneous voltage at switching instant is $2V_{dc}$ and thus the current ripple is relatively large.  

                \begin{figure}
                        \centering
                        \includegraphics[scale=0.4]{img/H-bridge_PWM_bipolar.png}
                        \caption{H-Bridge Bipolar PWM Switching \cite{kim17}}
                        \label{fig:PWM_bipolar}
                \end{figure}


        \item \textbf{Unipolar Switching: }This scheme triggers each switch independently by comparing the carrier wave to the 
                set voltage reference for points $V_{H}$ and $V_{L}$ which are the desired armature voltage and the negative of the 
                desired armature voltage respectively. If $V_{a}^{*}$ is greater than the carrier wave, then $T_{1}$ is switched ON 
                and $T_{3}$ OFF. Else, $T_{3}$ is ON and $T_{1}$ OFF. Independently, if  the carrier wave is less than $-V{a}^{*}$, 
                $T_{2}$ is ON and $T_{4}$ OFF, and vice versa as shown in Figure \ref{fig:PWM_unipolar}. 
                This scheme produces less current ripple as the maximum instanteneous voltage at any point is $V_{a}^{*}$.

                \begin{figure}
                        \centering
                        \includegraphics[scale=0.4]{img/H-bridge_PWM_unipolar.png}
                        \caption{H-Bridge Unipolar PWM Switching \cite{kim17}}
                        \label{fig:PWM_unipolar}
                \end{figure}


\end{itemize}


\pagebreak

\bibliographystyle{plain}
\bibliography{references} 

\end{document}